\documentclass[letterpaper,11pt]{article} % Copyright (c) 2019  Brian Schubert
\usepackage[margin=0.8in]{geometry}
\usepackage{amsmath}
\usepackage{amsthm}
\usepackage{tabularx}
\usepackage{fancyhdr}
\usepackage[table]{xcolor}

\thispagestyle{fancy}
\cfoot{\scriptsize Copyright \textcopyright \ 2019 Brian Schubert.}
\rfoot{\scriptsize Revised 2019-10-16}
\lhead{MATH 2341\\Differential Equations and Linear Algebra}
\rhead{Fall 2019 \\ Prof. Prasanth George}
\renewcommand{\headrulewidth}{0pt}	


\theoremstyle{definition}
\newtheorem*{definition*}{Definition}


\begin{document}
%    \begin{tabularx}{\textwidth}{Xr}
%        MATH 2341 & Fall 2019 \\ Differential Equations and Linear Algebra & Prasanth Geoge
%    \end{tabularx}
    \begin{center}
        \LARGE{\underline{Common Laplace Transforms}} \\
    \end{center}

    \begin{definition*}[Laplace Transform]
        Let $f$ be a continuous function of $t$, where $t \in [0, \infty)$. We define the Laplace Transform $\mathcal L\{f(t)\}$ by
        \begin{equation*}
            \mathcal L\{f(t)\} = \int_0^\infty e^{-st} f(t) \, \mathrm dt
        \end{equation*}
        where $s$ is the complex ``frequency" parameter.
        
        Remark 1. Note that $\mathcal L\{f(x)\}$ is a function of the variable $s$, and that $\mathcal L$ is linear with respect to $f$.
        
        Remark 2. Common alternative notation for the Laplace Transform of $f$ include $L[f(t)]$ and $F(s)$.
    \end{definition*}
    
    In the table below, let $Y(s)$ denote $\mathcal{L}\{y(t)\}$ and $F(s)$ denote $\mathcal{L}\{f(t)\}$.
    \vspace{-10pt}
    \renewcommand{\arraystretch}{2.3}
    \begin{table}[h]
        \large
        {\rowcolors{3}{white}{lightgray!25}
            \begin{tabularx}{0.95\textwidth}{
                    *{3}{>{\centering\arraybackslash}X}
                }
                \textbf{Case} & $\mathbf{f(t)}$ & $\mathbf{\mathcal{L}\{f(t)\}}$ \\
                \hline
                (1) & $k$ & \( \displaystyle \frac{k}{s}\) \\
                (2) & $t^n$ &  \( \displaystyle \frac{n!}{s^{n+1}} \) \\
                (3) & $e^{at}$ & \( \displaystyle \frac{1}{s-a}\) \\
                (4) & $t^n e^{at}$ & \( \displaystyle \frac{n!}{(s-a)^{n+1}} \) \\ % \frac{n!}{(s-a)^{n+1}} (n!) / (s-a)^{n+1}
                (5) & $\cos \left( kt \right)$ & \( \displaystyle \frac{s}{s^2 + k^2}\) \\
                (6) & $\sin \left( kt \right)$ & \( \displaystyle \frac{k}{s^2 + k^2}\) \\
                (7) & $e^{at}\cos \left( kt \right)$ & \( \displaystyle \frac{s-a}{(s-a)^2 + k^2} \) \\
                    % \frac{s-a}{(s-a)^2 + k^2} \left(s-a\right)/\left((s-a)^2 + k^2\right)
                (8) & $e^{at}\sin \left( kt \right)$ & \( \displaystyle\frac{k}{(s-a)^2 + k^2}\) \\
                    % \frac{k}{(s-a)^2 + k^2}  k/\left((s-a)^2 + k^2\right)
                (9) & $t \sin \left( kt \right)$ & \( \displaystyle \frac{2ks}{\left(s^2 + k^2\right)^2}\) \\
                    % \frac{2ks}{\left(s^2 + k^2\right)^2} \left(2ks\right)/\left(s^2 + k^2\right)^2
                (10) & $t \cos \left( kt \right)$ & \( \displaystyle \frac{s^2 - k^2}{\left(s^2 + k^2\right)^2}\) \\ 
                \hline
                    %\frac{\left(s^2 - k^2\right)}{\left(s^2 + k^2\right)^2} \left(s^2 - k^2\right) / \left(s^2 + k^2\right)^2
                (11) Step Function & $u(t-a)$ &  \( \displaystyle \frac{e^{-as}}{s}\) \\
                (12) Delta Function & $\delta(t-a)$ &  \( \displaystyle e^{-as}\) \\
                (13) 2nd Shifting Theorem & $u(t-a)f(t-a)$ &  \( \displaystyle e^{-as}F(s)\) \\
                (14) Derivative Transforms & $y'$ &  \( \displaystyle sY(s) - y(0) \) \\
                (15) Derivative Transforms & $y''$ &  \( s^2 Y(s) - sy(0) - y'(0 )\) \\
            \end{tabularx}
    }
    \end{table}
\end{document}